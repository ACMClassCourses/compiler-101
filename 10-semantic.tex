\documentclass{article}
\usepackage[utf8]{inputenc}
\usepackage[english]{babel}
\usepackage[letterpaper,top=2cm,bottom=2cm,left=3cm,right=3cm,marginparwidth=1.75cm]{geometry}
\usepackage{amsmath}
\usepackage{amssymb}
\usepackage{enumerate}
\usepackage{graphicx}
\usepackage{subcaption}
\usepackage{float}
\usepackage{wrapfig}
\usepackage[export]{adjustbox}
\usepackage[UTF8]{ctex}
\usepackage[colorlinks=true, allcolors=blue]{hyperref}
\usepackage{verbatim}


\title{语义检查(Semantic)}

\begin{document}

\maketitle

\section{Introduction}
\noindent
(先随便写写) \\
(中英混杂有点别扭??)\\
1. 基本信息:编译器作业整体分为semantic、codegen、optimizer三个部分,不限语言,推荐用Java完成。\\
2. 强烈建议认真阅读Yx的代码和tutorial.md。 \\
Yx是一个比Mx更加简单的语法,阅读Yx的实现对于代码的完成非常有帮助。\\
Yx的github地址:https://github.com/ZYHowell/Yx \\
3. 参考用书:(已下发) \\
4. 一些建议:多和同学、助教交流;没有头绪时多看看Yx和学长学姐的代码,有助于理解并提高效率;
切忌抄袭。\\


\section{Semantic}
\subsection{Introduction}
\noindent
在一个编译器的结构中,semantic阶段主要完成了词法分析(lexer)、语法分析(parser)、语义分析(semantic)的部分。
这一部分的评测要求是,正确地判断一段Mx代码是否存在词法、语法、语义上的错误。\\
lexer和parser的部分,我们将使用antlr4帮助我们完成;然后,我们将利用antlr4为我们生成的代码,构建AST(以下会有详解),并
进行semantic check。\\
接下来,我们将首先讲解对于semantic阶段工作的整体理解,再对具体的实现步骤逐步解析。

\subsection{Overview}
semantic部分的难点,主要在于如何构建AST。AST即abstract syntax tree,指抽象语法树。通俗地说,我们将源代转换成用树的结构来表示,
这棵树叫做AST。antlr4能够为我们完成lexer和parser,并提供接口便于我们提取其中的信息,我们希望能够自定义树的每个结点,并储存我们所需要的
信息,所以我们将继承antlr4生成的类,然后构建一棵属于我们的AST。\\

在一切工作开始之前,我们先根据Mx语言的规定,high level地了解一下我们的AST。 \\

一份Mx代码将被转换成一棵AST,这棵树从RootNode开始,随着树的层数的加深,
每层的结点所表示的单位逐渐变小。下面是基于Mx的一个AST层级的简单示意:
\begin{verbatim}
RootNode 
- VariableDef
- ClassDefination
    - VariableDefination
    - FuncDefination
- FunctionDefination - Statements - Expressions 
\end{verbatim}

Mx语言下,对这个结构进行一点简单的注解:\\
1. RootNode从全局作用域开始。Mx的全局作用域中只存在全局变量、类定义、全局函数三种类型。\\
2. 类定义中,只存在变量和函数。 \\
3. 一个函数包含多个statement(语句)。在AST中,一个函数中的每个statement都是这个函数节点的子节点。
statement有许多种类型,比如if语句、循环语句(for、while)等等。\\
4. expression(表达式)有许多种类型,比如基本表达式(primary)、常量表达式(constant)、赋值表达式(assignment)等等。 \\
5. 关于Mx所使用的statement和expression,Mx文档中均有明确的规定,概念模糊的同学可以先仔细阅读。 \\

如果你对以上结构感到费解,可以仔细阅读Yx库中的g4文件(Yx/src/parser/Yx.g4)辅助理解,或者继续阅读下一部分中对于如何构建AST的详细介绍,
准确地理解AST的结构将会大大提高效率。\\

\subsection{Practical Methods}

\subsubsection{Write the Grammar}

\subsubsection{The Parse Part in compiler}

\subsubsection{Design and Build the AST}

\subsubsection{Semantic Check}



\end{document}
