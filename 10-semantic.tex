\chapter{语义检查(Semantic)}
\noindent
(先写一点备注) \\
1. 强烈建议认真阅读Yx的代码和Tutorial.md。 \\
Yx是一个比Mx更加简单的语法,阅读Yx的实现对于代码的完成非常有帮助。\\
Yx的github地址:https://github.com/ZYHowell/Yx \\
2. 一些建议:多和同学、助教交流;没有头绪时多看看Yx和学长学姐的代码,有助于理解并提高效率;
切忌抄袭。\\

\section{Introduction}
在一个编译器的结构中,semantic阶段主要完成了词法分析(lexer)、语法分析(parser)、语义分析(semantic)的部分。
这一部分的评测要求是,正确地判断一段Mx代码是否存在词法、语法、语义上的错误。\\

lexer和parser的部分,我们将使用antlr4帮助我们完成;然后,我们将利用antlr4为我们生成的代码,构建AST(以下会有详解),并
进行semantic check。
接下来,我们将首先讲解对于semantic阶段工作的整体理解,再对具体的实现步骤逐步解析。

\section{Overview}
semantic部分的难点,主要在于如何构建AST。AST即abstract syntax tree,指抽象语法树。通俗地说,我们将源代转换成用树的结构来表示,
这棵树叫做AST。antlr4能够为我们完成lexer和parser,并提供接口便于我们提取其中的信息,我们希望能够自定义树的每个结点,并储存需要的
信息,所以我们将继承antlr4生成的类,然后构建一棵属于我们的AST。\\

在一切工作开始之前,我们先根据Mx语言的规定,整体了解一下我们的AST。 \\

一份Mx代码将被转换成一棵AST,这棵树从RootNode开始,随着树的层数的加深,
每层的结点所表示的单位逐渐变小。下面是基于Mx的一个AST层级的简单示意:
\begin{verbatim}
RootNode 
- VariableDef
- ClassDefinition
    - VariableDefinition
    - FuncDefinition
- FunctionDefinition - Statements - Expressions 
\end{verbatim}

Mx语言下,对这个结构进行一点简单的注解:\\
1. RootNode从全局作用域开始。Mx的全局作用域中只存在全局变量、类定义、全局函数三种类型。\\
2. 类定义中,只存在变量和函数。 \\
3. 一个函数包含多个statement(语句)。在AST中,一个函数中的每个statement都是这个函数节点的子节点。
statement有许多种类型,比如if语句、循环语句(for、while)等等。\\
4. expression(表达式)有许多种类型,比如基本表达式(primary)、常量表达式(constant)、赋值表达式(assignment)等等。 \\
5. 关于Mx所使用的statement和expression,Mx文档中均有明确的规定,概念模糊的同学可以先仔细阅读。 \\

如果你对以上结构感到费解,可以仔细阅读Yx库中的g4文件(Yx/src/parser/Yx.g4)辅助理解,或者继续阅读下一部分中对于如何构建AST的详细介绍,
准确地理解AST的结构将会大大提高效率。\\

\section{Practical Methods}
以下是semantic部分的具体实现过程。

\subsection{About antlr4}
antlr4是一个强大的解析器生成器。在我们的实现中,我们需要先写一个Mx.g4文件,用来描述Mx的语法,然后运行ANTLR命令,
生成MxLexer.java, MxParser.java等文件。\\

antlr4的安装、运行等等内容,请阅读《antlr4权威指南》第一部分的相应内容(pdf已下发,17页开始)。\\

antlr4的其他详细内容,请参照《antlr4权威指南》。

\subsection{Write the Grammar}
关于g4文件如何书写,《antlr4权威指南》中已经给出了详细的指导,可以结合Yx的代码理解。\\

下面简单介绍一下g4文件各部分内容的意义:\\
1. \\

提醒: \\
1. 关于优先级问题 \\
2. 使用 "\#"  \\
3. 使用 "<assoc=right>"


\subsection{The Parse Part in compiler}
1. 生成文件解释 \\
对完成的g4文件,运行antlr4后,将生成一系列文件,下面对几个文件进行一个简单地介绍。 \\


2. 调用生成代码完成lexer和parser \\
如何在我们的代码中调用antlr4生成的文件来完成lexer和paerser?Yx的main文件已经给出了示例。
下面的代码是Yx的部分代码:\\
\begin{verbatim}
    YxLexer lexer = new YxLexer(CharStreams.fromStream(input));     // 1
    lexer.removeErrorListeners();                                   // 2
    lexer.addErrorListener(new YxErrorListener());                  // 3    
    YxParser parser = new YxParser(new CommonTokenStream(lexer));   // 4
    parser.removeErrorListeners();                                  // 5
    parser.addErrorListener(new YxErrorListener());                 // 6
    ParseTree parseTreeRoot = parser.program();                     // 7
\end{verbatim}
注解: \\


3. 自定义error \\
antlr4在lexer和parser的过程中,能够发现词法、语法错误,而具体的报错函数可以自定义。
这里可以参考Yx对于YxErrorListener类的处理。

\subsection{Design and Build the AST}
\subsubsection{About the Tree}

\subsubsection{ASTNode}

\subsubsection{Scope}

\subsubsection{ASTBuilder}


\subsection{Semantic Check}
1. SymbolCollector \\

2. SemanticChecker \\




