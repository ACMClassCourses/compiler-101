\chapter{课程要求}
% \begin{introduction}
%     \item TBA
% \end{introduction}

编译器设计课程在 ACM 班有非常悠久的历史,该课程要求学生自主完成一个从源代码到汇编代码的编译器,引导学生直观理解系统执行二进制代码的过程并掌握一系列代码级别优化方法。
与传统编译原理课程不同的是,本课程不会提供任何的已有框架。课程设计目标要求自主设计数据结构(自主设计抽象语法树、中间表达,自主探索语言特定的优化策略)。

\begin{remark}
    在本课程中,允许使用 \texttt{antlr} 等前端分析库,但是不得使用现有的编译库。
\end{remark}

一段程序的执行可以抽象为两个问题,即下一步要执行的指令是哪条以及需要访问的数据资源是哪些。我们将前者抽象为控制流 (Control Flow),后者抽象为数据流 (Data Flow)。
数据流可以简单理解为访问内存、寄存器,而控制流可以按照执行的流程粗略抽象分为3种:顺序、分支、循环。
编译的过程可以理解为让计算机理解一段高级语言的数据流和控制流。

以下是实现编译器的步骤的概述:

\begin{enumerate}
    \item 词法分析:第一步是将源代码转换为一个词素流,在这个阶段需要指定基本的词素单元。
    \item 句法分析:根据生成的词素流分析源代码的结构并且按照语言的语法规则转换为层次化的结构,生成解析树或抽象语法树(AST)。
    \item 语义分析:解析后,进行语义分析以检查程序的正确性。这包括类型检查、名称解析、作用域分析和其他语义检查。在这个阶段需要构建一个符号表并执行各种检查,以确保程序的正确性。
    \item 中间表示:在这个阶段,将AST转换为中间表示 (IR)。IR是程序的一个抽象表示,更容易分析和优化。常见的 IR 大多使用虚拟寄存器的抽象。
    \item 代码优化:根据生成的中间表示对程序进行优化,其中关键一步是分配寄存器(如果采用虚拟寄存器抽象需要转换为真实的寄存器)。此外,优化还包括死代码消除、函数内联等。
    \item 代码生成:优化后,生成目标汇编代码。该步骤需要将源语言/中间表达映射到目标体系结构的相应汇编指令,需要指令选择,以生成高效的汇编代码。
\end{enumerate}

为了方便大家分阶段开发,我们将会把作业划分为三个阶段。语义检查阶段(对应1、2、3)、目标代码生成阶段(阶段4、6)、寄存器分配(阶段5)。各个阶段的要求如下:
\begin{table}[!ht]
    \resizebox{\textwidth}{!}{\begin{tabular}{c|c|c}
        \hline
                 & 目标                                                                          & 要求                                                                            \\ \hline
        语义检查阶段   & \begin{tabular}[c]{@{}c@{}}将源代码转换为一个具有语义信息的\\ 抽象结构并实现对语言的语法检查。\end{tabular} & \begin{tabular}[c]{@{}c@{}}通过编译器可以识别存在语法错\\ 误的代码。\end{tabular}                \\ \hline
        目标代码生成阶段 & \begin{tabular}[c]{@{}c@{}}将高级语言转换为汇编语言并且实现\\ 简单的指令选择。\end{tabular}         & \begin{tabular}[c]{@{}c@{}}可以生成对应的可终止汇编代码\\ (不对性能做出要求)。\end{tabular}          \\ \hline
        寄存器分配    & \begin{tabular}[c]{@{}c@{}}在目标代码或中间表达等结构的基础\\ 上实现高效的寄存器分配算法。\end{tabular}   & \begin{tabular}[c]{@{}c@{}}在测评程序集上与标准Clang编译的\\ 结果进行性能比对(按照周期数)。\end{tabular} \\ \hline
        \end{tabular}}
\end{table}
