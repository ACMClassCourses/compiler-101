\chapter{编译器设计总述}
% \begin{introduction}
%     \item TBA
% \end{introduction}

编译器设计课程在ACM班有非常悠久的历史,该课程要求学生自主完成一个从源代码到汇编代码的编译器,引导学生直观理解系统执行二进制代码的过程并掌握一系列代码级别优化方法。
与传统编译原理课程不同的是,本课程不会提供任何的已有框架,且设计目标主要是自主设计数据结构(自主设计抽象语法树、中间表达,自主探索语言特定的优化策略)。

\begin{remark}
    在本课程中,允许使用\texttt{antlr}等前端分析库,但是不得使用现有的编译库。
\end{remark}

一段程序的执行可以抽象为两个问题,即下一步要执行的指令是哪条以及需要访问的数据资源是哪些。我们将前者抽象为控制流(Control Flow),后者抽象为数据流(Data Flow)。
数据流可以简单理解为访问内存、寄存器,而控制流可以按照执行的流程粗略抽象分为3种:顺序、分支、循环。
遗憾的是,高级语言对控制流和数据流的感知都较差。

以下是实现编译器的步骤的概述:

\begin{enumerate}
    \item 词法分析(标记化):第一步是将源代码分解成一系列标记。这个过程称为词法分析或标记化。您需要定义语言标记的规则,并编写一个词法分析器来识别和分类标记。
    \item 语法分析(解析):一旦有了标记,您需要分析它们的结构并强制执行语言的语法规则。这一步称为语法分析或解析。您可以使用递归下降解析或解析器生成器(如Yacc或Bison)等技术来生成解析树或抽象语法树(AST),表示源代码的结构。
    \item 语义分析:解析后,进行语义分析以检查程序的正确性。这包括类型检查、名称解析、作用域分析和其他语义检查。您需要构建一个符号表并执行各种检查,以确保程序的正确性。
    \item 中间表示(IR)生成:在这个阶段,将AST转换为中间表示(IR)。IR是程序的一个抽象表示,更容易分析和优化。常见的IR形式包括三地址代码、控制流图或抽象堆栈机器。
    \item 优化:一旦有了IR,可以应用各种优化来提高生成代码的性能或大小。优化技术包括常量折叠、死代码消除、循环优化等。这一步是可选的,但可以极大地提高编译代码的效率。
    \item 代码生成:优化后,生成目标汇编代码。这涉及将源语言的高级结构映射到目标体系结构的相应汇编指令。您需要处理内存管理、寄存器分配和指令选择,以生成高效的汇编代码。
    \item 汇编:生成的汇编代码需要组装成机器代码。这一步涉及将汇编指令转换为可由目标处理器执行的二进制机器代码。您可以使用汇编器或特定于目标体系结构的汇编工具链。
    \item 链接:如果程序由多个源文件组成或使用外部库,需要将生成的目标文件链接在一起生成最终的可执行文件。链接器解析不同目标文件之间的引用,并生成一个完整的可执行文件。
\end{enumerate}