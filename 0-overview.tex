\chapter{编译器设计总述}
\begin{introduction}
    \item TBA
\end{introduction}

编译器设计课程在ACM班有非常悠久的历史,该课程要求学生自主完成一个从源代码到汇编代码的编译器,引导学生直观理解系统执行二进制代码的过程并掌握一系列代码级别优化方法。
与传统编译原理课程不同的是,本课程不会提供任何的已有框架,且设计目标主要是自主设计数据结构(自主设计抽象语法树、中间表达,自主探索语言特定的优化策略)。

\begin{remark}
    在本课程中,允许使用\texttt{antlr}等前端分析库,但是不得使用现有的编译库。
\end{remark}